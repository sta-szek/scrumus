\documentclass[a4paper,11pt]{article}
\usepackage{latexsym}
\usepackage[MeX]{polski}
\usepackage[utf8]{inputenc}
\usepackage{url}
\usepackage{listings}
\usepackage{color}
\usepackage[pdftex]{graphicx}

\definecolor{mygreen}{rgb}{0,0.6,0}
\definecolor{mygray}{rgb}{0.5,0.5,0.5}
\definecolor{mymauve}{rgb}{0.58,0,0.82}

\lstset{ %
	backgroundcolor=\color{white},  
	basicstyle=\footnotesize,        % the size of the fonts that are used for the code
	breakatwhitespace=true,          % sets if automatic breaks should only happen at whitespace
	breaklines=false,                % sets automatic line breaking
	captionpos=b,                    % sets the caption-position to bottom
	commentstyle=\color{mygreen},    % comment style
	deletekeywords={},            	 % if you want to delete keywords from the given language
	escapeinside={\%*}{*)},          % if you want to add LaTeX within your code
	extendedchars=true,              % lets you use non-ASCII characters; for 8-bits encodings only, does not work with UTF-8
	frame=single,                    % adds a frame around the code
	keepspaces=true,                 % keeps spaces in text, useful for keeping indentation of code (possibly needs columns=flexible)
	keywordstyle=\color{blue},       % keyword style
	language=Java,                 	 % the language of the code
	otherkeywords={},           	 % if you want to add more keywords to the set
	numbers=left,                    % where to put the line-numbers; possible values are (none, left, right)
	numbersep=-10pt,                 % how far the line-numbers are from the code
	numberstyle=\tiny\color{mygray}, % the style that is used for the line-numbers
	rulecolor=\color{black},         % if not set, the frame-color may be changed on line-breaks within not-black text (e.g. comments (green here))
	showspaces=false,                % show spaces everywhere adding particular underscores; it overrides 'showstringspaces'
	showstringspaces=false,          % underline spaces within strings only
	showtabs=false,                  % show tabs within strings adding particular underscores
	stepnumber=1,                    % the step between two line-numbers. If it's 1, each line will be numbered
	stringstyle=\color{mymauve},     % string literal style
	tabsize=2,                       % sets default tabsize to 2 spaces
	title=\lstname                   % show the filename of files included with \lstinputlisting; also try caption instead of title
}

\newenvironment{italicquote}
{\begin{quote}\itshape}
{\end{quote}}

\author{Piotr Joński}
\title{Historyjki użytkownika}
%\frenchspacing - Wyłączenie większych odstępów na końcu zdań

\begin{document}
% Autor i tytuł
\maketitle
% Spis tresci
\tableofcontents
\newpage

\section{Historyjki użytkownika}
Jednym ze sposobów na prowadzenie dokumentacji projektu, jak i zbioru wymagań jest utrzymywanie rejestru historyjek użytkownika. Historyjki użytkownika są częścią zwinnych metody prowadzenia projektu. Jako że wytwarzanemu systemowi towarzyszy metodyka Scrum, nie mogło tutaj zabraknąć tego elementu.

\begin{italicquote}
	Historyjka jest jednostką funkcjonalnościc w projektach XP. Pokazujemy postęp prac, dostarczając przetestowany i zintegrowany kod, który składa się na implementację danej historyjki. Historyjka powinna być zrozumiała i wartościowa dla klientów, testowalna przez programistów i na tyle mała, żeby programiści mogli zaimplementować sześć historyjek w takcie jednej iteracji\footnote{K. Beck, M. Fowler, \textit{Planning Extreme Programming}, Addison-Wesley, Bostion 2000, s.42}.
\end{italicquote}

Historyjki mogą mieć wiele wzorców. W tej pracy są stosowane dwa z nich:
\begin{itemize}
	\item Jako \textit{\textless typ użytkownika \textgreater} mogę \textit{\textless nazwa zadania\textgreater}.
	\item Jako \textit{\textless typ użytkownika \textgreater} mogę \textit{\textless nazwa zadania\textgreater} w celu \textit{\textless cel \textgreater}\cite{SCRUM}
\end{itemize} 

W projektowanym systemie występują cztery rodzaje użytkowników (administrator, product owner, scrum master oraz developer) z różnymi uprawnieniami. Teraz szczegółowo zostaną opisane uprawnienia każdego z nich w postaci \textit{user stories}.
	\newline
	\newline
	\textbf{Administrator}
	\begin{enumerate}		
		\item Jako administrator mogę utworzyć projekt w celu dodania go do systemu.
		\item Jako administrator mogę usunąć projekt w celu usunięcia z systemu oraz powiązanych z nim elementów t.j. sprint, story, backlog oraz inne. Operacja usuwania odbywa się kaskadowo.
		\item Jako administrator mogę tworzyć nowych użytkowników w celu dodania ich do systemu.
		\item Jako administrator mogę usuwać użytkowników w celu usunięcia z systemu.
		\item Jako administrator mogę nadawać dowolne uprawnienia wszystkim użytkownikom.
		\item Jako administrator mogę tworzyć nowe grupy w celu dodania ich do systemu.
		\item Jako administrator mogę dodawać użytkowników do grup oraz usuwać z nich w celu modyfikacji zespołu.		 
	\end{enumerate}		
	\textbf{Product owner}
	\begin{enumerate}		
		\item Jako product owner mogę tworzyć nowe zadania w projekcie, w celu dodania ich do backlogu.
		\item Jako product owner mogę nadawać priorytety poszczególnym zadaniom w projekcie.
		\item Jako product owner mam wgląd w cały projekt, do którego jestem przypisany.
	\end{enumerate}	
	\textbf{Scrum master}
	\begin{enumerate}		
		\item Co "specjalnego" scrum master może robić w moim systemie?
	\end{enumerate}
	\textbf{Developer}
	\begin{enumerate}		
		\item Jako developer mogę tworzyć i edytować zadania.
		\item Jako developer mogę dodawać komentarze do zadań oraz retrospektyw.
		\item Jako developer mogę usuwać swoje komentarze do zadań oraz retrospektyw.
		\item Jako developer mogę edytować zadania.
		\item Jako developer mogę edytować swój profil.
	\end{enumerate}
	

\newpage
\begin{thebibliography}{99}
	\bibitem{JSF} JavaServer Faces i Eclipse Galileo. Tworzenie aplikacji WWW \emph{Andrzej Marciniak} Helion 2010
	\bibitem{JSF_CORE} Core JavaServer Faces. Wydanie II \emph{David Geary, Cay S. Horstmann} Helion 2008
	\bibitem{EJB_3.0} Enterprise JavaBeans 3.0 \emph{Bill Burke \& Richard Monson-Haefel} Helion 2007
	\bibitem{JBOSS_7} JBoss AS 7. Tworzenie aplikacji \emph{Francesco Marchioni} Helion 2014
	\bibitem{J2EE} Core J2EE. Wzorce projektowe \emph{Deepak Alur, John Crupi, Dan Malks} Helion 2004
	\bibitem{SCRUM} Scrum. O zwinnym zarządzaniu projektami. Wydanie II rozszerzone \emph{Mariusz Chrapko} Helion 2015
	\bibitem{JAVA} Java. Kompedium programisty. Wydanie VIII \emph{Herbert Schildt} Helion 2012
	\bibitem{CI} Zwinne wytwarzanie oprogramowania. Najlepsze zasady, wzorce i praktyki \emph{Robert C. Martin} Helion 2015
	\bibitem{GIT} Git. Rozproszony system kontroli wersji \emph{Włodzimierz Gajda} Helion 2013
	\bibitem{CLEAN_CODE_MASTER} Mistrz czystego kodu. Kodeks postępowania profesjonalnych programistów \emph{Robert C. Martin} Helion 2013
	\bibitem{CLEAN_CODE} Czysty kod. Podręcznik dobrego programisty \emph{Robert C. Martin} Helion 2010
	\bibitem{WZORCE} Rusz głową! Wzorce projektowe \emph{Eric Freeman, Elisabeth Freeman, Bert Bates, Kathy Sierra} Helion 2011
	\bibitem{REFACTOR} Refaktoryzacja do wzorców projektowych \emph{Joshua Kerievsky} Helion 2005
	
\end{thebibliography}

\end{document}