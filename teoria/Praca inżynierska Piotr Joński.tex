\documentclass[a4paper,11pt]{book}
\usepackage{latexsym}
\usepackage[MeX]{polski}
\usepackage[utf8]{inputenc}
\usepackage{url}
\usepackage{rotating}
\usepackage{listings}
\usepackage{color}
\usepackage[pdftex]{graphicx}
\usepackage{footnote}
\usepackage{packages/zmienne}	
\usepackage{packages/strona_tytulowa}
\usepackage{packages/oswiadczenie}
\usepackage{packages/karta_pracy}
\usepackage{extsizes}							%wiecej rozmiaróww czcionek
\usepackage[a4paper,left=3.5cm,right=2.5cm,top=2.5cm,bottom=2.5cm]{geometry}
\usepackage{tocloft}								% format spisu treści
\usepackage{array}								% lepiej wyglądające tabelki
\usepackage[format=hang,labelsep=period,labelfont={bf,small},textfont=small]{caption}
\usepackage{floatflt}							% ładniejsze opisywanie obrazków tekstem
\usepackage{subfig}								% możliwość wstawiania figur w kolumnach
\usepackage{graphicx}							% do obsługi grafiki
\usepackage{here}									% wymuszanie położenia figury w danym miejscu
\usepackage{url}									% adresy internetowe
\usepackage{enumerate}							% modyfikowanie list wyliczeniowych np \begin{enumerate}[(a)]...
\usepackage{multirow}							% do tabel
\usepackage{slantsc}
\usepackage[T1]{fontenc}
\usepackage{lmodern}\normalfont %to load T1lmr.fd
\usepackage{algorithm}
\usepackage{algorithmic}
\usepackage{amsmath}
\usepackage{amssymb}
\usepackage[pdftex,usenames,dvipsnames]{color}

\usepackage{dashrule}
\usepackage{fancyhdr} 							% do stopki i nagłówka
\usepackage{calc}
\usepackage{packages/zmienne}				
\usepackage{longtable}							% do podziału tabel na wiele stron

\usepackage{indentfirst}
\floatname{algorithm}{Algorytm}
\usepackage[section]{placeins}
\usepackage{nomencl}
\makenomenclature
\usepackage{makeidx}
\makeindex
\renewcommand{\nomname}{Spis ważniejszych symboli}

\definecolor{mygreen}{rgb}{0,0.6,0}
\definecolor{mygray}{rgb}{0.5,0.5,0.5}
\definecolor{mymauve}{rgb}{0.58,0,0.82}

\lstset{ %
	backgroundcolor=\color{white},  
	basicstyle=\footnotesize,        % the size of the fonts that are used for the code
	breakatwhitespace=true,          % sets if automatic breaks should only happen at whitespace
	breaklines=false,                % sets automatic line breaking
	captionpos=b,                    % sets the caption-position to bottom
	commentstyle=\color{mygreen},    % comment style
	deletekeywords={},            	 % if you want to delete keywords from the given language
	escapeinside={\%*}{*)},          % if you want to add LaTeX within your code
	extendedchars=true,              % lets you use non-ASCII characters; for 8-bits encodings only, does not work with UTF-8
	frame=single,                    % adds a frame around the code
	keepspaces=true,                 % keeps spaces in text, useful for keeping indentation of code (possibly needs columns=flexible)
	keywordstyle=\color{blue},       % keyword style
	language=Java,                 	 % the language of the code
	otherkeywords={@Data, @NoArgsConstructor, @Entity, @NamedQueries, @Table, @NamedQuery},           	 % if you want to add more keywords to the set
	numbers=left,                    % where to put the line-numbers; possible values are (none, left, right)
	numbersep=-10pt,                 % how far the line-numbers are from the code
	numberstyle=\tiny\color{mygray}, % the style that is used for the line-numbers
	rulecolor=\color{black},         % if not set, the frame-color may be changed on line-breaks within not-black text (e.g. comments (green here))
	showspaces=false,                % show spaces everywhere adding particular underscores; it overrides 'showstringspaces'
	showstringspaces=false,          % underline spaces within strings only
	showtabs=false,                  % show tabs within strings adding particular underscores
	stepnumber=1,                    % the step between two line-numbers. If it's 1, each line will be numbered
	stringstyle=\color{mymauve},     % string literal style
	tabsize=2,                       % sets default tabsize to 2 spaces
	title=\lstname                   % show the filename of files included with \lstinputlisting; also try caption instead of title
}

\newenvironment{italicquote}
{\begin{quote}\itshape}
{\end{quote}}

\author{Piotr~Joński}
\kierunek{Informatyka}
\specjalnosc{Sieciowe systemy informatyczne}
\grupa{431 IDZ}
\title{System zarządzania projektami dedykowany metodyce Scrum}
\tytulAngielski{Project management system dedicated to Scrum methodology}
\uczelnia{Uniwersytet Zielonogórski}
\wydzial{Wydział Informatyki, Elektrotechniki i Automatyki}
\praca{Praca dyplomowa}
\promotor{dr inż. Andrzej~Marciniak}
\konsultant{} 
\miasto{Zielona Góra}
\miesiac{luty}
\rok{2016}
\dzien{10} 			
\mm{02}
\podjecieTematu{18.03.2015} 


\celPracy{Celem pracy jest wytworzenie systemu, który będzie umożliwiał intuicyjne zarządzanie projektem.}

\numZakres{4}

\zakresI{Opracowanie wymagań funkcjonalnych i niefunkcjonalnych dla projektowanego systemu,}
\zakresII{Wytworzenie projektu i implementacja systemu,}
\zakresIII{Wykonanie testów aplikacji,}
\zakresIV{Wytworzenie dokumentacji i zredagowanie części opisowej pracy.}

\pagenumbering{roman}

\makeatletter
    \def\numberline#1{\hb@xt@\@tempdima{#1.\hfil}}                      %kropki w spisie treœci
    \renewcommand*\@seccntformat[1]{\csname the#1\endcsname.\enspace}   %kropki w treści dokumentu
\makeatother

\makeatother
% ------------------------------------------------------------------------
% Definicje
% ------------------------------------------------------------------------
\def\nonumsection#1{%
    \section*{#1}%
    \addcontentsline{toc}{section}{#1}%
    }
\def\nonumsubsection#1{%
    \subsection*{#1}%
    \addcontentsline{toc}{subsection}{#1}%
    }
\reversemarginpar %umieszcza notki po lewej stronie, czyli tam gdzie jest wiêcej miejsca
\def\notka#1{%
    \marginpar{\footnotesize{#1}}%
    }
%\def\mathcal#1{%
%    \mathscr{#1}%
%    }

\newcommand{\myemptypage}{ \newpage  \thispagestyle{empty}~\newpage}

%-------------------------------------------------------------------------
% stopka i nagłówek
%-------------------------------------------------------------------------
\setlength{\headheight}{15pt}

\pagestyle{fancy}
\renewcommand{\chaptermark}[1]{\markboth{#1}{}}
\renewcommand{\sectionmark}[1]{\markright{#1}{}}

\fancyhf{}
\fancyhead[LE,RO]{\thepage}
\fancyhead[RE]{\textit{\nouppercase{\leftmark}}}
\fancyhead[LO]{\textit{\nouppercase{\rightmark}}}

\fancypagestyle{plain}{ %
\fancyhf{}
\renewcommand{\headrulewidth}{0pt}
\renewcommand{\footrulewidth}{0pt}}

% ------------------------------------------------------------------------
% Inne
% ------------------------------------------------------------------------
\frenchspacing
\setlength{\parskip}{3pt}           	%odstêp pomiêdzy akapitami
%\linespread{1.49}                    	%odstêp pomiêdzy liniami (interlinia)
\setcounter{tocdepth}{3}
\setcounter{secnumdepth}{3}


% ------------------------------------------------------------------------
% Polskie podpisy
% ------------------------------------------------------------------------
\renewcommand{\figurename}{Rys.}
\renewcommand{\tablename}{Tab.}

% ------------------------------------------------------------------------
% Bibliografia
% ------------------------------------------------------------------------
\bibliographystyle{unsrt}					% kolejnoœæ wed³óg u¿ycia
%\bibliographystyle{plain}					% kolejnoϾ alfabetyczna



%==========================================================================================
% Deklaracja fontow kapitalikowych z kodowaniem T1
%==========================================================================================
\DeclareFontShape{T1}{lmr}{bx}{sc} { <-> ssub * cmr/bx/sc }{}
\DeclareFontShape{T1}{lmr}{bx}{scit}{<-> ssub * cmr/bx/scsl}{}
%==========================================================================================
% Inne deklaracje
%==========================================================================================
\renewcommand*{\lstlistlistingname}{Spis listingów}
\newcommand{\specialcell}[2][c]{%
		\begin{tabular}[#1]{@{}c@{}}#2\end{tabular}}
	
	

\begin{document}

\newgeometry{left=2cm, right=2cm, top=1cm,bottom=2cm,headsep=1cm}
\thispagestyle{empty}
\kartapracy
\myemptypage

\newgeometry{left=3.5cm,right=2.5cm,top=2.5cm,bottom=2.5cm}	
\linespread{1.49}
\thispagestyle{empty}
\stronatytulowa

\normalsize
\oswiadczenie
\newpage

\subsection*{Streszczenie}


CO TU DAĆ?


\vspace{1cm}
\noindent\textbf{słowa kluczowe:} praca dyplomowa, skład komputerowy, formatowanie dokumentu.
\myemptypage

%========================================================================================
% Spis tresci, spis tabel i~rysunków
%========================================================================================
%spis tresci
\tableofcontents
\newpage
%\myemptypage
%spis rysunków
\listoffigures
\newpage
%\myemptypage
%spis tabel
\listoftables
\newpage
%\myemptypage

%========================================================================================
% Licznik Stron
%========================================================================================
\newcounter{licznikStron}
\setcounter{licznikStron}{\value{page}}
\setcounter{licznikStron}{1}
\pagenumbering{arabic}
\setcounter{page}{\value{licznikStron}}

%Rozdziały
\chapter{Wstęp}
\section{Wprowadzenie i przegląd literatury}

Co tu dać?

\section{Cel i zakres pracy}
Celem pracy było wytworzenie systemu do zarządzania projektami dedykowanego metody Scrum. Obecnie na rynku jest wiele zarówno darmowych jak i płatnych narzędzi, które oferują możliwość prowadzenia projektów rożnymi metodykami. W zakres pracy wchodzą:
\begin{enumerate}
	\item zapoznanie się z literaturą tematu,
	\item opracowanie założeń projektu,
	\item spis wymagań funkcjonalnych i niefunkcjonalnych systemu,
	\item implementacja wszystkich funkcjonalności oraz usunięcie powstałych błędów,
	\item przetestowanie systemu ,
	\item wytworzenie części opisowej pracy.
	
\end{enumerate}


\section{Struktura pracy}
Opis rozdziałów zostanie uzupełniony na końcu pracyitemize
\chapter{Wprowadzenie}
\section{Szkic problemu}

Szybki rozwój w dziedzinie informatyki spowodował wzrost zapotrzebowania na systemy, które pomogłyby rozwiązywać problemy kwestii organizacyjnej projektów. W dzisiejszych czasach zarządzanie projektami jest szerokim zagadnieniem, a na jego temat powstało wiele publikacji. Autorzy proponują takie rozwiązania dot. zarządzania projektami jak:
\begin{enumerate}
	\item scrum,
	\item lean Softrware Development,
	\item feature Driven Development,
	\item test Driven Develipment.
\end{enumerate}

Nie jest to wyczerpująca lista metodyk - istnieje wiele innych metod tworzenia oprogramowania, jednak opis wszystkich wykracza po za zakres tej pracy. W mojej pracy dyplomowej poruszam temat systemu do zarządzania projektami, który jest dedykowany metodyce Scrum. Nie bez powodu wybrana została właśnie ta zwinna metodyka - jest to na chwilę obecną najbardziej popularna metoda wytwarzania oprogramowania, a umiejętności pracy wraz z nią są pożądane przez większość pracodawców na całym świecie.

Scrum jest zwinną metodyką wytwarzania oprogramowania, której początki sięgają połowy lat 80. Aby prawidłowo przedstawić omawiany temat należy zapoznać się z pojęciami, które są nieodłącznym elementem tej metody:
\begin{enumerate}
	\item \textbf{product owner} - osoba odpowiedzialna za projekt,
	\item \textbf{scrum master} - osoba, która pomaga zespołowi w organizacji czasu pracy a także rozwiązuje powstałe problemu,
	\item \textbf{zespół developerski} - zespół programistów wspólnie pracujący nad wytworzeniem produktu,
	\item \textbf{backlog produktu} - jest to zbiór zadań / funkcjonalności wchodzące w skład wytwarzanego projektu,
	\item \textbf{sprint} - interwały czasowe, w których realizowane są zadania. Bezpośrednio przed każdym sprintem występuje planowanie, czyli wybieranie funkcjonalności do zrealizowania w danym sprincie. Bezpośrednio po sprincie powinna wystąpić retrospektywa oraz demo produktu, które jest często pomijane przez wiele zespołów,
	\item \textbf{story} - najmniejsza jednostka podlegająca ocenie (wg. wybranej przez zespół skali). 
	\item \textbf{task} - zadania, które mogą być powiązane bezpośrednio ze story lub projektem.
	\item \textbf{retrospektywa} - wydarzenie, które ma miejsce na koniec sprintu. W tym momencie zespół developerski spisuje wszystkie wady i zalety minionego sprintu oraz wspólnie z scrum masterem starają się rozwiązać zaistniałe problemy.
\end{enumerate} 

\section{Przegląd dostępnych narzędzi}
Wytwarzając ten system skupiłem się, aby wpasował się on w wybraną przeze mnie metodykę oraz rozwiązywał szereg mankamentów, które napotkałem podczas korzystania z obecnych już rozwiązań. W celu lepszego zrozumienia problemów postanowiłem opisać kilka z nich.

Pierwszym i niewątpliwie największym problemem tego typu systemów jest to, że są one płatne, przez co nie wszystkich na nie stać. Niektóre są darmowe, lecz tylko dla projektów typu open source, co nie sprawdza się podczas pracy nad wspólnymi projektami w firmie lub np. pracą dyplomową.

Kolejną wadą jest to, że nie są one proste w instalacji. Często występują jako potężne programy, przez co nie można ich bezpłatnie hostować w chmurze gdyż potrzebują dużo płatnych zasobów. 

Ostatnim, jednak nie mniej ważnym problemem jest ich czytelność. Oferują one szereg zbędnej zwykłemu użytkownikowi - programiście - funkcjonalności, która jest przydatna tylko w określonych warunkach. Taki ogrom zakładek i okienek powoduje często zamęt i niezrozumienie wśród programistów.


Poniższa tabela przedstawia 
------------------------------------------------------------------------------------------------------

Celem pracy było wytworzenie systemu do zarządzania projektami dedykowanego metody Scrum. Obecnie na rynku jest wiele zarówno darmowych jak i płatnych narzędzi, które oferują możliwość prowadzenia projektów rożnymi metodykami. Mają one jednak pewne wady:
\begin{enumerate}
	\item Darmowe narzędzia nie są wspierane, przez co ich rozwój stanął w miejscu wiele lat temu
	\item Część z nich jest darmowa tylko dla 10 użytkowników, co powoduje, że nie są przydatne w większych firmach
	\item Większość jest darmowa tylko dla open source, co wyklucza możliwość użycia ich w firmie
	\item Posiadają przerost formy nad treścią - docelowy użytkownik - programista musi przedzierać się przez gąszcz funkcjonalności oraz poświęcać czas na system, zamiast na swoją pracę
\end{enumerate}

Wytworzony przeze mnie system ma na celu sprostać wadom obecnego oprogramowania oraz spełniać następujące wymagania:
\begin{enumerate}
	\item Łatwość instalacji - do tego celu zostało wykorzystane oprogramowanie docker, które wspiera szybkie wytwarzanie oprogramowania
	\item Przejrzysty interfejs użytkownika - został osiągnięty dzięki open sourcowej bibliotece PrimeFaces
	\item Bez zbędnej funkcjonalności - zostały zaimplementowane tylko obligatoryjne funkcje tego typu systemu oraz niewielka ilość udogodnień
\end{enumerate}



\section{Historyjki użytkownika}
Jednym ze sposobów na prowadzenie dokumentacji projektu, jak i zbioru wymagań jest utrzymywanie rejestru historyjek użytkownika. Historyjki użytkownika są częścią zwinnych metody prowadzenia projektu. Jako że wytwarzanemu systemowi towarzyszy metodyka Scrum, nie mogło tutaj zabraknąć tego elementu.

\begin{italicquote}
	Historyjka jest jednostką funkcjonalności w projektach XP. Pokazujemy postęp prac, dostarczając przetestowany i zintegrowany kod, który składa się na implementację danej historyjki. Historyjka powinna być zrozumiała i wartościowa dla klientów, testowalna przez programistów i na tyle mała, żeby programiści mogli zaimplementować sześć historyjek w takcie jednej iteracji\footnote{K. Beck, M. Fowler, \textit{Planning Extreme Programming}, Addison-Wesley, Boston 2000, s.42}.
\end{italicquote}

Historyjki mogą mieć wiele wzorców. W tej pracy są stosowane dwa z nich:
\begin{itemize}
	\item Jako \textit{\textless typ użytkownika \textgreater} mogę \textit{\textless nazwa zadania\textgreater}.
	\item Jako \textit{\textless typ użytkownika \textgreater} mogę \textit{\textless nazwa zadania\textgreater} w celu \textit{\textless cel \textgreater}\cite{SCRUM}
\end{itemize} 

W projektowanym systemie występują cztery rodzaje użytkowników (administrator, product owner, scrum master oraz developer) z różnymi uprawnieniami. Teraz szczegółowo zostaną opisane uprawnienia każdego z nich w postaci \textit{user story}. Warto pamiętać, że w systemie występuje uproszczony model użytkowników. Dodatkowo każdy z nich jest developerem, a co za tym idzie, może wykonywać akcje użytkownika developer oraz własne.
	\newline
	\newline
	\textbf{Administrator}
	\begin{enumerate}	
		\item Jako administrator mogę tworzyć nowych użytkowników w celu dodania ich do systemu.
		\item Jako administrator mogę usuwać użytkowników z systemu.
		\item Jako administrator mogę dodawać użytkowników do grup oraz usuwać z nich w celu modyfikacji zespołu.
		\item Jako administrator mogę nadawać dowolne uprawnienia wszystkim użytkownikom.		
		\item Jako administrator mogę utworzyć projekt w celu dodania go do systemu.
		\item Jako administrator mogę usunąć projekt w celu usunięcia z systemu oraz powiązanych z nim elementów t.j. sprint, story, backlog oraz inne. Operacja usuwania odbywa się kaskadowo.
		\item Jako administrator mogę modyfikować projekt.		
		\item Jako administrator mogę tworzyć nowe grupy w celu dodania ich do systemu.		
		\item Jako administrator mogę dodawać i usuwać grupy z projektów w celu przydzielenia uprawnień.
		\item Jako administrator mogę tworzyć, usuwać oraz edytować statusy zadań.
		\item Jako administrator mogę tworzyć, usuwać oraz edytować priorytety zadań.
	\end{enumerate}		
	\textbf{Product owner}
	\begin{enumerate}		
		\item Jako product owner mogę tworzyć nowe zadania w projekcie, w celu dodania ich do backlogu.
		\item Jako product owner mogę nadawać priorytety poszczególnym zadaniom w projekcie.
		\item Jako product owner mam wgląd w cały projekt, do którego jestem przypisany.
	\end{enumerate}	
	\textbf{Scrum master}
	\begin{enumerate}		
		\item Jako scrum master mogę tworzyć nowe sprinty w projekcie.
		\item Jako scrum master mogę tworzyć nowe story w sprintach.
		\item Jako scrum master mogę dodawać i usuwać zadania ze story.
		\item Jako scrum master mogę tworzyć retrospektywy.
		\item Jako scrum master mam wgląd we wszystkie projekty zespołów, w których jestem scrum masterem.
	\end{enumerate}
	\textbf{Developer}
	\begin{enumerate}		
		\item Jako developer mogę tworzyć i edytować zadania.
		\item Jako developer mogę dodawać komentarze do zadań oraz retrospektyw.
		\item Jako developer mogę usuwać swoje komentarze do zadań oraz retrospektyw.
		\item Jako developer mogę edytować swój profil.
	\end{enumerate}
	

\newpage
\begin{thebibliography}{99}
	\bibitem{JSF} JavaServer Faces i Eclipse Galileo. Tworzenie aplikacji WWW \emph{Andrzej Marciniak} Helion 2010
	\bibitem{JSF_CORE} Core JavaServer Faces. Wydanie II \emph{David Geary, Cay S. Horstmann} Helion 2008
	\bibitem{EJB_3.0} Enterprise JavaBeans 3.0 \emph{Bill Burke \& Richard Monson-Haefel} Helion 2007
	\bibitem{JBOSS_7} JBoss AS 7. Tworzenie aplikacji \emph{Francesco Marchioni} Helion 2014
	\bibitem{J2EE} Core J2EE. Wzorce projektowe \emph{Deepak Alur, John Crupi, Dan Malks} Helion 2004
	\bibitem{SCRUM} Scrum. O zwinnym zarządzaniu projektami. Wydanie II rozszerzone \emph{Mariusz Chrapko} Helion 2015
	\bibitem{JAVA} Java. Kompedium programisty. Wydanie VIII \emph{Herbert Schildt} Helion 2012
	\bibitem{CI} Zwinne wytwarzanie oprogramowania. Najlepsze zasady, wzorce i praktyki \emph{Robert C. Martin} Helion 2015
	\bibitem{GIT} Git. Rozproszony system kontroli wersji \emph{Włodzimierz Gajda} Helion 2013
	\bibitem{CLEAN_CODE_MASTER} Mistrz czystego kodu. Kodeks postępowania profesjonalnych programistów \emph{Robert C. Martin} Helion 2013
	\bibitem{CLEAN_CODE} Czysty kod. Podręcznik dobrego programisty \emph{Robert C. Martin} Helion 2010
	\bibitem{WZORCE} Rusz głową! Wzorce projektowe \emph{Eric Freeman, Elisabeth Freeman, Bert Bates, Kathy Sierra} Helion 2011
	\bibitem{REFACTOR} Refaktoryzacja do wzorców projektowych \emph{Joshua Kerievsky} Helion 2005	
\end{thebibliography}

\end{document}