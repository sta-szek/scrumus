\documentclass[a4paper,11pt]{book}
\usepackage{latexsym}
\usepackage[MeX]{polski}
\usepackage[utf8]{inputenc}
\usepackage{url}
\usepackage{rotating}
\usepackage{listings}
\usepackage{color}
\usepackage[pdftex]{graphicx}
\usepackage{footnote}
\usepackage{packages/zmienne}	
\usepackage{packages/strona_tytulowa}
\usepackage{packages/oswiadczenie}
\usepackage{packages/karta_pracy}
\usepackage{extsizes}							%wiecej rozmiaróww czcionek
\usepackage[a4paper,left=3.5cm,right=2.5cm,top=2.5cm,bottom=2.5cm]{geometry}
\usepackage{tocloft}								% format spisu treści
\usepackage{array}								% lepiej wyglądające tabelki
\usepackage[format=hang,labelsep=period,labelfont={bf,small},textfont=small]{caption}
\usepackage{floatflt}							% ładniejsze opisywanie obrazków tekstem
\usepackage{subfig}								% możliwość wstawiania figur w kolumnach
\usepackage{graphicx}							% do obsługi grafiki
\usepackage{here}									% wymuszanie położenia figury w danym miejscu
\usepackage{url}									% adresy internetowe
\usepackage{enumerate}							% modyfikowanie list wyliczeniowych np \begin{enumerate}[(a)]...
\usepackage{multirow}							% do tabel
\usepackage{slantsc}
\usepackage[T1]{fontenc}
\usepackage{lmodern}\normalfont %to load T1lmr.fd
\usepackage{algorithm}
\usepackage{algorithmic}
\usepackage{amsmath}
\usepackage{amssymb}
\usepackage[pdftex,usenames,dvipsnames]{color}

\usepackage{dashrule}
\usepackage{fancyhdr} 							% do stopki i nagłówka
\usepackage{calc}
\usepackage{packages/zmienne}				
\usepackage{longtable}							% do podziału tabel na wiele stron

\usepackage{indentfirst}
\floatname{algorithm}{Algorytm}
\usepackage[section]{placeins}
\usepackage{nomencl}
\makenomenclature
\usepackage{makeidx}
\makeindex
\renewcommand{\nomname}{Spis ważniejszych symboli}

\definecolor{mygreen}{rgb}{0,0.6,0}
\definecolor{mygray}{rgb}{0.5,0.5,0.5}
\definecolor{mymauve}{rgb}{0.58,0,0.82}

\lstset{ %
	backgroundcolor=\color{white},  
	basicstyle=\footnotesize,        % the size of the fonts that are used for the code
	breakatwhitespace=true,          % sets if automatic breaks should only happen at whitespace
	breaklines=false,                % sets automatic line breaking
	captionpos=b,                    % sets the caption-position to bottom
	commentstyle=\color{mygreen},    % comment style
	deletekeywords={},            	 % if you want to delete keywords from the given language
	escapeinside={\%*}{*)},          % if you want to add LaTeX within your code
	extendedchars=true,              % lets you use non-ASCII characters; for 8-bits encodings only, does not work with UTF-8
	frame=single,                    % adds a frame around the code
	keepspaces=true,                 % keeps spaces in text, useful for keeping indentation of code (possibly needs columns=flexible)
	keywordstyle=\color{blue},       % keyword style
	language=Java,                 	 % the language of the code
	otherkeywords={},           	 % if you want to add more keywords to the set
	numbers=left,                    % where to put the line-numbers; possible values are (none, left, right)
	numbersep=-10pt,                 % how far the line-numbers are from the code
	numberstyle=\tiny\color{mygray}, % the style that is used for the line-numbers
	rulecolor=\color{black},         % if not set, the frame-color may be changed on line-breaks within not-black text (e.g. comments (green here))
	showspaces=false,                % show spaces everywhere adding particular underscores; it overrides 'showstringspaces'
	showstringspaces=false,          % underline spaces within strings only
	showtabs=false,                  % show tabs within strings adding particular underscores
	stepnumber=1,                    % the step between two line-numbers. If it's 1, each line will be numbered
	stringstyle=\color{mymauve},     % string literal style
	tabsize=2,                       % sets default tabsize to 2 spaces
	title=\lstname                   % show the filename of files included with \lstinputlisting; also try caption instead of title
}

\newenvironment{italicquote}
{\begin{quote}\itshape}
{\end{quote}}

\author{Piotr~Joński}
\kierunek{Informatyka}
\specjalnosc{Sieciowe systemy informatyczne}
\grupa{431IDZ}
\title{System zarządzania projektami dedykowany metodyce Scrum}
\tytulAngielski{Project management system dedicated to Scrum methodology}
\uczelnia{Uniwersytet Zielonogórski}
\wydzial{Wydział Informatyki, Elektrotechniki i Automatyki}
\praca{Praca dyplomowa}
\promotor{dr inż. Andrzej~Marciniak}
\konsultant{} 
\miasto{Zielona Góra}
\miesiac{02}
\rok{2016}
\dzien{10} 			
\mm{02}
\podjecieTematu{01.01.2000} 


\celPracy{PRZEPISAĆ Z KARTY DYPLOMOWEJ}

\numZakres{3}

\zakresI{Zakres pracy 1}
\zakresII{Zakres pracy 2}
\zakresIII{Zakres pracy 3}

\pagenumbering{roman}

\makeatletter
    \def\numberline#1{\hb@xt@\@tempdima{#1.\hfil}}                      %kropki w spisie treœci
    \renewcommand*\@seccntformat[1]{\csname the#1\endcsname.\enspace}   %kropki w treści dokumentu
\makeatother

\makeatother
% ------------------------------------------------------------------------
% Definicje
% ------------------------------------------------------------------------
\def\nonumsection#1{%
    \section*{#1}%
    \addcontentsline{toc}{section}{#1}%
    }
\def\nonumsubsection#1{%
    \subsection*{#1}%
    \addcontentsline{toc}{subsection}{#1}%
    }
\reversemarginpar %umieszcza notki po lewej stronie, czyli tam gdzie jest wiêcej miejsca
\def\notka#1{%
    \marginpar{\footnotesize{#1}}%
    }
%\def\mathcal#1{%
%    \mathscr{#1}%
%    }

\newcommand{\myemptypage}{ \newpage  \thispagestyle{empty}~\newpage}

%-------------------------------------------------------------------------
% stopka i nagłówek
%-------------------------------------------------------------------------
\setlength{\headheight}{15pt}

\pagestyle{fancy}
\renewcommand{\chaptermark}[1]{\markboth{#1}{}}
\renewcommand{\sectionmark}[1]{\markright{#1}{}}

\fancyhf{}
\fancyhead[LE,RO]{\thepage}
\fancyhead[RE]{\textit{\nouppercase{\leftmark}}}
\fancyhead[LO]{\textit{\nouppercase{\rightmark}}}

\fancypagestyle{plain}{ %
\fancyhf{}
\renewcommand{\headrulewidth}{0pt}
\renewcommand{\footrulewidth}{0pt}}

% ------------------------------------------------------------------------
% Inne
% ------------------------------------------------------------------------
\frenchspacing
\setlength{\parskip}{3pt}           	%odstêp pomiêdzy akapitami
%\linespread{1.49}                    	%odstêp pomiêdzy liniami (interlinia)
\setcounter{tocdepth}{3}
\setcounter{secnumdepth}{3}


% ------------------------------------------------------------------------
% Polskie podpisy
% ------------------------------------------------------------------------
\renewcommand{\figurename}{Rys.}
\renewcommand{\tablename}{Tab.}

% ------------------------------------------------------------------------
% Bibliografia
% ------------------------------------------------------------------------
\bibliographystyle{unsrt}					% kolejnoœæ wed³óg u¿ycia
%\bibliographystyle{plain}					% kolejnoϾ alfabetyczna



%==========================================================================================
% Deklaracja fontow kapitalikowych z kodowaniem T1
%==========================================================================================
\DeclareFontShape{T1}{lmr}{bx}{sc} { <-> ssub * cmr/bx/sc }{}
\DeclareFontShape{T1}{lmr}{bx}{scit}{<-> ssub * cmr/bx/scsl}{}
%==========================================================================================
% Inne deklaracje
%==========================================================================================

\newcommand{\specialcell}[2][c]{%
		\begin{tabular}[#1]{@{}c@{}}#2\end{tabular}}