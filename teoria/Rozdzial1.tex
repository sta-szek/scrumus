\chapter{Wstęp}
\section{Wprowadzenie}

Szybki rozwój w dziedzinie informatyki spowodował wzrost zapotrzebowania na systemy, które pomogłyby rozwiązywać problemy kwestii organizacyjnej projektów. W dzisiejszych czasach zarządzanie projektami jest szerokim zagadnieniem, a na jego temat powstało wiele publikacji. Autorzy proponują takie rozwiązania dot. zarządzania projektami jak:
\begin{enumerate}
	\item Scrum,
	\item Lean Softrware Development,
	\item Feature Driven Development,
	\item Test Driven Develipment.
\end{enumerate}

Nie jest to wyczerpująca lista metodyk - istnieje wiele innych metod tworzenia oprogramowania, jednak opis wszystkich wykracza po za zakres tej pracy. W mojej pracy dyplomowej poruszam temat systemu do zarządzania projektami, który jest dedykowany metodyce Scrum. Nie bez powodu wybrana została właśnie ta zwinna metodyka - jest to na chwilę obecną najbardziej popularna forma wytwarzania oprogramowania, a umiejętności pracy wraz z nią są pożądane przez większość pracodawców na całym świecie.

Obecnie na rynku istnieje mnóstwo narzędzi do tego typu zadań, a liczba oferowanych szkoleń z zakresu zarządzania projektami oraz metodyki Scrum nie ma końca. Coraz więcej firm decyduje się na zakup drogich systemów wspomagających proces zarządzania projektami. Ale czy zakup takiego programu jest konieczny? W mojej pracy przedstawię system, który bez żadnych przeszkód mógłby być stosowany w niewielkich zespołach, czy firmach, które nie zamierzają wydawać fortuny na profesjonalne aplikacje.

\section{Czym właściwie jest scrum}
Scrum jest zwinną metodyką wytwarzania oprogramowania, której początki sięgają połowy lat 80. Polega ona na przyrostowym wytwarzaniu produktu oraz stałej komunikacji z klientem bądź też jego reprezentantem - właścicielem produktu (\textit{ang. product owner}). Zespół deweloperski ({\textit{ang. team}}) podejmuje się przygotowania wybranych przez siebie funkcjonalności, których wybór może być zależny od właściciela produktu - to on ustala priorytety. Jednak w celu zminimalizowania ryzyka "wyższej władzy", czyli narzucaniu zbyt wysokich wymagań i nieosiągalnych terminów, co często czynią właściciele produktu, została wprowadzona kolejna rola - scrum master, którą ciężko jest przetłumaczyć na język polski (z tego względu w dalszej części pracy będę posługiwał się właśnie tym terminem). To on "chroni" zespół przed przepracowaniem i rozwiązuje napotykane po drodze problemy. Aby prawidłowo przedstawić omawiany temat należy zapoznać się z kilkoma pojęciami, które są nieodłącznym elementem tej metody:
\begin{itemize}
	\item \textbf{właściciel produktu} (\textit{ang. product owner}) - osoba odpowiedzialna za projekt,
	\item \textbf{scrum master} - osoba, która pomaga zespołowi w organizacji czasu pracy a także rozwiązuje powstałe problemu,
	\item \textbf{zespół deweloperski} (\textit{ang. team}) - zespół programistów wspólnie pracujący nad wytworzeniem produktu,
	\item \textbf{rejestr produktu} (\textit{ang. backlog}) - jest to zbiór zadań / funkcjonalności wchodzące w skład wytwarzanego projektu,
	\item \textbf{interwał} (\textit{ang. sprint}) - interwały czasowe, w których realizowane są zadania. Bezpośrednio przed każdym sprintem występuje planowanie, czyli wybieranie funkcjonalności do zrealizowania w danym sprincie. Bezpośrednio po sprincie powinna wystąpić retrospektywa oraz demo produktu, które jest często pomijane przez wiele zespołów,
	\item \textbf{historyjka} (\textit{ang. story}) - najmniejsza jednostka podlegająca ocenie (wg. wybranej przez zespół skali). 
	\item \textbf{zadanie} (\textit{ang. task}) - zadania, które mogą być powiązane bezpośrednio ze story lub projektem.
	\item \textbf{retrospektywa} (\textit{ang. retrospective}) - wydarzenie, które ma miejsce na koniec sprintu. W tym momencie zespół deweloperski spisuje wszystkie wady i zalety minionego sprintu oraz wspólnie z scrum masterem starają się rozwiązać zaistniałe problemy. Jest to jeden z ważniejszych elementów Scruma.
\end{itemize} 

\section{Przegląd dostępnych narzędzi}
Wytwarzając ten system skupiłem się, aby wpasował się on w wybraną przeze mnie metodykę oraz rozwiązywał szereg mankamentów, które napotkałem podczas korzystania z obecnych już rozwiązań. W celu lepszego zrozumienia problemów postanowiłem opisać kilka z nich.

Pierwszym i niewątpliwie największym problemem tego typu systemów jest to, że są one płatne, przez co nie wszystkich na nie stać. Niektóre są darmowe, lecz tylko dla projektów typu open source, co nie sprawdza się podczas pracy nad wspólnymi projektami w firmie lub np. pracą dyplomową.

Kolejną wadą jest to, że nie są one proste w instalacji. Często występują jako potężne programy, przez co nie można ich bezpłatnie hostować w chmurze gdyż potrzebują dużo płatnych zasobów. 

Ostatnim, jednak nie mniej ważnym problemem jest ich czytelność. Oferują one szereg zbędnej zwykłemu użytkownikowi - programiście - funkcjonalności, która jest przydatna tylko w określonych warunkach. Taki ogrom zakładek i okienek powoduje często zamęt i niezrozumienie wśród programistów.

Aby zobrazować wagę powyższych problemów zostały one zestawione w poniższej tabeli porównującej wybrane systemy wraz z systemem, który został wytworzony w ramach pracy dyplomowej. Nadałem mu nazwę \textbf{scrumus}, która ma charakter gry słownej - scrum us - w wolnym tłumaczeniu "wyskramuj nas".
\begin{table}[h!]
	\caption{Porównanie wybranych systemów zarządzania projektami}
	\centering
	\begin{tabular}{|c|c|c|c|c|}
		\hline
		\multirow{2}{*}{Cecha systemu} & \multicolumn{4}{c|}{Wybrane systemy} \\\cline{2-5} & JIRA & Bitbucket & GitHub & scrumus\\
		\hline
		Darmowy & do 10 użyt. & do 5 użyt. & open source & tak\\
		\hline
		Hosting & wew. / zew. & zew. & zew. & wew. / zew.\\
		\hline
		Czas instalacji & ok. 5 minut & nd. & nd. & ok. 2 minuty\\
		\hline
		Wymagania & baza danych & nd. & nd. & docker\\
		\hline
		Cechy & \specialcell{issue tracker\\ wiki} & \specialcell{issue tracker\\ wiki} & \specialcell{issue tracker\\ wiki} & \specialcell{issue tracker\\ wersja lite}\\
		\hline
		
	\end{tabular}
	\label{tabela:porownanie_systemow}
\end{table}

\section{Rozwiązanie problemu}

Celem pracy było wytworzenie systemu do zarządzania projektami dedykowanego metody Scrum. Obecnie na rynku jest wiele zarówno darmowych jak i płatnych narzędzi przedstawionych w tabeli \ref{tabela:porownanie_systemow}, które oferują możliwość prowadzenia projektów rożnymi metodykami. Wcześniej wymienione wady skłoniły mnie do opracowania własnego systemu, który ma za zadanie rozwiązać wspomniane problemy w następujący sposób:
\begin{enumerate}
	\item łatwość instalacji - do tego celu zostało wykorzystane oprogramowanie docker, które wspiera błyskawiczne uruchamianie systemów,
	\item przejrzysty interfejs użytkownika, który został osiągnięty dzięki darmowej bibliotece PrimeFaces,
	\item bez zbędnej funkcjonalności - zostały zaimplementowane tylko obligatoryjne funkcje tego typu systemu oraz niewielka ilość udogodnień.
\end{enumerate}

\section{Cel i zakres pracy}
Poza wyżej wymienionymi wymaganiami, jakie są stawiane systemowi scrumus, w zakres pracy wchodzą także:
\begin{enumerate}
	\item zapoznanie się z literaturą tematu,
	\item opracowanie założeń projektu,
	\item spis wymagań funkcjonalnych i niefunkcjonalnych systemu,
	\item implementacja wszystkich funkcjonalności oraz usunięcie powstałych błędów,
	\item przetestowanie systemu ,
	\item wytworzenie części opisowej pracy.
	
\end{enumerate}


\section{Struktura pracy}
Opis rozdziałów zostanie uzupełniony na końcu pracyitemize