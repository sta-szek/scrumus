\chapter{Analiza biznesowa problemu i założenia projektowe}

\section{Role i użytkownicy systemu}
W projektowanym systemie występują cztery rodzaje użytkowników:

\begin{enumerate}
	\item administrator - posiada on największe uprawnienia w systemie,
	\item właściciel produktu - odzwierciela rolę właściciela produktu w Scrumie,
	\item scrum master - odzwierciela rolę scrum mastera w Scrumie,
	\item deweloper - posiada najmniejsze uprawnienia, jest częścią zespołu deweloperskiego.
\end{enumerate} 

Na tym etapie warto wspomnieć, że każdy użytkownik jest jest deweloperem. Każdy projekt może mieć tylko jednego właściciela produktu, a każdy z nich może być product ownerem tylko raz. Dodatkowo zespół deweloperski ma przydzielonego tylko jednego scrum mastera, przy czym scrum master może być przypisany do wielu zespołów jednocześnie.

\section{Wymagania funkcjonalne}
Prezentowany przeze mnie system ma pewne założenia oraz wymagania. W tym rozdziale zajmę się opisem wymagań funkcjonalnych. 

We wcześniejszym akapicie zostały omówione role w systemie. Każda z tych ról ma pewne uprawnienia lub restrykcje. Każda taka cecha zostanie przedstawiona jako wymaganie funkcjonalne prezentowanego systemy.

Jednym ze sposobów na prowadzenie dokumentacji projektu, jak i zbioru wymagań jest utrzymywanie rejestru historyjek użytkownika. Historyjki użytkownika są częścią zwinnych metody prowadzenia projektu. Jako że wytwarzanemu systemowi towarzyszy metodyka Scrum, nie mogło tutaj zabraknąć tego elementu.

\begin{italicquote}
	Historyjka jest jednostką funkcjonalności w projektach XP. Pokazujemy postęp prac, dostarczając przetestowany i zintegrowany kod, który składa się na implementację danej historyjki. Historyjka powinna być zrozumiała i wartościowa dla klientów, testowalna przez programistów i na tyle mała, żeby programiści mogli zaimplementować sześć historyjek w takcie jednej iteracji\footnote{K. Beck, M. Fowler, \textit{Planning Extreme Programming}, Addison-Wesley, Boston 2000, s.42}.
\end{italicquote}

Historyjki mogą mieć wiele wzorców. W tej pracy są stosowane dwa z nich:
\begin{itemize}
	\item Jako \textit{\textless typ użytkownika\textgreater} mogę \textit{\textless nazwa zadania\textgreater}.
	\item Jako \textit{\textless typ użytkownika\textgreater} mogę \textit{\textless nazwa zadania\textgreater} w celu \textit{\textless cel\textgreater}\cite{SCRUM}
\end{itemize} 


\textbf{Administrator}
\begin{enumerate}	
	\item jako administrator mogę tworzyć nowych użytkowników w celu dodania ich do systemu,
	\item jako administrator mogę usuwać użytkowników z systemu z wyjątkiem siebie samego,
	\item jako administrator mogę dodawać użytkowników do zespołu w celu modyfikacji zespołu,
	\item jako administrator mogę usuwać użytkowników z zespołu w celu modyfikacji zespołu,
	\item jako administrator mogę nadać uprawnienia administratora dowolnemu użytkownikowi,
	\item jako administrator mogę odebrać uprawnienia administratora dowolnemu administratorowi,
	\item jako administrator mogę przypisać właściciela produktu do projektu,
	\item jako administrator mogę usunąć właściciela produktu z projektu,
	\item jako administrator mogę utworzyć projekt w celu dodania go do systemu,
	\item jako administrator mogę usunąć projekt w celu usunięcia z systemu oraz powiązanych z nim elementów t.j. sprint, story, backlog oraz inne. Operacja usuwania odbywa się kaskadowo,
	\item jako administrator mogę modyfikować projekt,
	\item jako administrator mogę tworzyć nowe zespoły w celu dodania ich do systemu,
	\item jako administrator mogę dodawać zespoły do projektów w celu przydzielenia uprawnień,
	\item jako administrator mogę usuwać zespoły z projektów w celu odebrania uprawnień,
	\item jako administrator mogę przypisać scrum mastera do zespołu,
	\item jako administrator mogę usunąć scrum mastera z zespołu,
	\item jako administrator mogę tworzyć, usuwać oraz edytować statusy zadań,
	\item jako administrator mogę tworzyć, usuwać oraz edytować priorytety zadań,
	\item jako administrator mogę tworzyć, usuwać oraz edytować typy zadań.
\end{enumerate}		
\textbf{Product owner}
\begin{enumerate}		
	\item TODO
\end{enumerate}	
\textbf{Scrum master}
\begin{enumerate}		
	\item TODO
\end{enumerate}
\textbf{Developer}
\begin{enumerate}		
	\item jako deweloper mogę tworzyć zadania,
	\item jako deweloper mogę usuwać zadania,
	\item jako deweloper mogę dodawać komentarze do zadań,
	\item jako deweloper mogę dodawać komentarze do retrospektyw,
	\item jako deweloper mogę edytować swój profil,
	\item jako deweloper mogę zmienić swoje hasło,
	\item jako deweloper mogę przeglądać profile innych użytkowników,
	\item jako deweloper mogę przeglądać wszystkie projekty, do których jestem przypisany,
	\item jako deweloper mogę przypisać zadanie do dowolnego użytkownika,
	
\end{enumerate}

\section{Wymagania funkcjonalne}
Prezentowany przeze mnie system ma pewne założenia oraz wymagania. W tym rozdziale zajmę się opisem wymagań funkcjonalnych. 